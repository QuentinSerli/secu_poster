%%%%%%%%%%%%%%%%%%%%%%%%%%%%%%%%%%%%%%%%%
% Dreuw & Deselaer's Poster
% LaTeX Template
% Version 1.0 (11/04/13)
%
% Created by:
% Philippe Dreuw and Thomas Deselaers
% http://www-i6.informatik.rwth-aachen.de/~dreuw/latexbeamerposter.php
%
% This template has been downloaded from:
% http://www.LaTeXTemplates.com
%
% License:
% CC BY-NC-SA 3.0 (http://creativecommons.org/licenses/by-nc-sa/3.0/)
%
%%%%%%%%%%%%%%%%%%%%%%%%%%%%%%%%%%%%%%%%%

%----------------------------------------------------------------------------------------
%	PACKAGES AND OTHER DOCUMENT CONFIGURATIONS
%----------------------------------------------------------------------------------------

\documentclass[final,hyperref={pdfpagelabels=false}]{beamer}

\usepackage[orientation=portrait,size=a0, scale=1]{beamerposter} % Use the beamerposter package for laying out the poster with a portrait orientation and an a0 paper size

\usetheme{I6pd2} % Use the I6pd2 theme supplied with this template

\usepackage[english]{babel} % English language/hyphenation

\usepackage{amsmath,amsthm,amssymb,latexsym} % For including math equations, theorems, symbols, etc

%\usepackage{times}\usefonttheme{professionalfonts}  % Uncomment to use Times as the main font
%\usefonttheme[onlymath]{serif} % Uncomment to use a Serif font within math environments

\boldmath % Use bold for everything within the math environment

\usepackage{booktabs} % Top and bottom rules for tables

\usepackage[normalem]{ulem}
\graphicspath{{figures/}} % Location of the graphics files

\usecaptiontemplate{\small\structure{\insertcaptionname~\insertcaptionnumber: }\insertcaption} % A fix for figure numbering

%----------------------------------------------------------------------------------------
%	TITLE SECTION 
%----------------------------------------------------------------------------------------

\title{\huge Top 10 vulnérabilités Webapp\\comment ne pas se faire p0wn} % Poster title

\author{Quentin Mallet} % Author(s)

\institute{Serli} % Institution(s)

%----------------------------------------------------------------------------------------
%	FOOTER TEXT
%----------------------------------------------------------------------------------------

\newcommand{\leftfoot}{https://www.owasp.org} % Left footer text

\newcommand{\rightfoot}{quentin.mallet@serli.com} % Right footer text

%----------------------------------------------------------------------------------------

\begin{document}

\addtobeamertemplate{block end}{}{\vspace*{2ex}} % White space under blocks

\begin{frame}[t] % The whole poster is enclosed in one beamer frame

\begin{columns}[t] % The whole poster consists of two major columns, each of which can be subdivided further with another \begin{columns} block - the [t] argument aligns each column's content to the top

\begin{column}{.02\textwidth}\end{column} % Empty spacer column

\begin{column}{.465\textwidth} % The first column

%----------------------------------------------------------------------------------------
%	OBJECTIVES
%----------------------------------------------------------------------------------------

\begin{block}{1: Injections}

	\begin{columns}[T]

		\begin{column}{.01\textwidth}
		\end{column}
		\begin{column}{.115\textwidth} % The second subdivided column within the first main column
			\includegraphics[scale=1.35]{syringe.png}
		\end{column}

		\begin{column}{.875\textwidth}
			\uline{\textbf{Symptomes}}
			\begin{itemize}
			\item Les utilisateurs peuvent soumettre des données au service.
			\item Ces données ne sont pas filtrées. \textbf{par exemple:} Robert'); DROP TABLE students;\# 
			\end{itemize}
		\end{column}
	\end{columns}
	\begin{itemize}
		\item Ces données sont directement utilisées dans des requêtes. \textbf{par exemple:} INSERT INTO students(name) VALUES('\$student\_name')
		\item \textbf{BOOM} INSERT INTO students(name) VALUES('Robert'); DROP TABLE students;\#\textit{')}
	\end{itemize}

	\begin{columns}[T]
    \begin{column}{.01\textwidth}
    \end{column}
	\begin{column}{.45\textwidth} % The second subdivided column within the first main column
	\vfill
	\uline{\textbf{Effets}}
	\begin{itemize}
		\item Compromission totale de la base de données.
		\item Prise de contrôle du serveur.
		\item ... et bien d'autres choses affreuses.
	\end{itemize}
		\end{column}

	\begin{column}{.45\textwidth} % The second subdivided column within the first main column
		\vfill
		\uline{\textbf{Prévention}}
		\begin{itemize}
			\item Requêtes paramétrisées ("INSERT INTO x(a,b,c) VALUES(?,?,?)")
			\item Whitelisting des entrées 
			\item Echapper les caractères spéciaux
		\end{itemize}
	\end{column}
	\end{columns}

\end{block}

%----------------------------------------------------------------------------------------
%	INTRODUCTION
%----------------------------------------------------------------------------------------
            
\begin{block}{2: Authentification cassée}

	\begin{columns}[T]

		\begin{column}{.01\textwidth}
		\end{column}
		\begin{column}{.115\textwidth} % The second subdivided column within the first main column
			\includegraphics[scale=2.3]{rimlock.png}
		\end{column}

		\begin{column}{.875\textwidth}
			\uline{\uline{\textbf{Symptomes}}}
			\begin{itemize}
				\item L'authentification n'est pas chiffrée (on peut intercepter les mots de passe)
			\end{itemize}
		\end{column}
	\end{columns}
	\begin{itemize}
		\item Le token de session est facilement deviné (nom d'utiliateur, timestamp)
		\item Mots de passe faibles acceptés (P4ssword)
		\item ID de session exposé dans les URLs
		\item Pas de blocage après X tentatives de connexion
		\item Mots de passe stockés en clair dans les bases de données.
	\end{itemize}
	\begin{columns}[T]
        \begin{column}{.01\textwidth}
		\end{column}
        \begin{column}{0.495\textwidth}
			\vfill
			\uline{\textbf{Effets}}
			\begin{itemize}
				\item Vol d'identité
				\item Fuite de données sensibles
				\item Prise de contrôle du serveur par un acteur malveillant.
			\end{itemize}
			\vfill
		\end{column}
		\begin{column}{0.495\textwidth}
			\vfill
			\uline{\textbf{Prévention}}
			\begin{itemize}
				\item Obligation d'utiliser des mots de passe forts.
				\item Tokens de sessions aléatoires
				\item Cooldown/blockage de l'authentification
			\end{itemize}
		\end{column}
	\end{columns}


\end{block}

%----------------------------------------------------------------------------------------
%	MATERIALS
%----------------------------------------------------------------------------------------

\begin{block}{Exposition de données sensibles}
	\begin{columns}[T]

		\begin{column}{.01\textwidth}
		\end{column}
		\begin{column}{.115\textwidth} % The second subdivided column within the first main column
			\includegraphics[scale=1.35]{binoculars.png}
		\end{column}

		\begin{column}{.875\textwidth}
			\uline{\uline{\textbf{Symptomes}}}
			\begin{itemize}
				\item Données transmises sans chiffrement (http,smtp, ftp,...)
				\item Données stockées sans chiffrement
				\item Algorithmes de chiffrement obsolets 
			\end{itemize}
		\end{column}
	\end{columns}
	\begin{itemize}
		\item Backups stockés sans chiffrement 
		\item Clés de chiffrement à mot de passe faible
	\end{itemize}
	\begin{columns}[T]
		\begin{column}{.01\textwidth}
		\end{column}
		\begin{column}{0.495\textwidth}
			\vfill
			\uline{\textbf{Effets}}
			\begin{itemize}
				\item Vol d'identité
				\item Fuite de données sensibles
				\item Perte de contrôle du SI en cas de brêche des bases de données d'administration
			\end{itemize}
			\vfill
		\end{column}
		\begin{column}{0.495\textwidth}
			\vfill
			\uline{\textbf{Prévention}}
			\begin{itemize}
				\item Classification des données manipulées par sensibilité et stockage différencié.
				\item Ne pas stocker les données qui peuvent être facilement resoumises par l'utilisateur, ou utiliser une tokenization/troncation.
				\item Ne pas conserver en cache des données sensibles.
				\item Chiffrer les données au repos.
				\item Chiffrer les données en transit (TLS par éxemple)
				\item Hacher et saler les mots de passe (Argon2, scrypt, bcrypt, pbkdf2).
			\end{itemize}
		\end{column}
	\end{columns}



\end{block}

\begin{block}{Exposition de données sensibles}
	\begin{columns}[T]

		\begin{column}{.12\textwidth} % The second subdivided column within the first main column
			\includegraphics[scale=1.35]{xml.png}
		\end{column}

		\begin{column}{.88\textwidth}
			\uline{\uline{\textbf{Symptomes}}}
			\begin{itemize}
				\item L'application accepte du XML ou permets de l'upload de XML
                \item N'importe lequel des processeurs XML a le
                    \href{https://www.w3schools.com/xml/xml_dtd_intro.asp}{\uline{\textit{DTD}}}
                    activé
				\item Si l'application utilise du SAML pour l'authentification
                    Single Sign On, vérifier qu'elle n'est pas vulnérable.
			\end{itemize}
		\end{column}
	\end{columns}
    \begin{itemize}
        \item Si l'application utilise SOAP $<$ 1.2 vérifier qu'elle n'est
            pas vulnérable.
    \end{itemize}
	\begin{columns}[T]
		\begin{column}{.01\textwidth}
		\end{column}
		\begin{column}{0.495\textwidth}
			\vfill
			\uline{\textbf{Effets}}
			\begin{itemize}
                \item Attaque par déni de service (notamment l'attaque \textit{Billion
                    Laughs}
				\item Fuite de données sensibles
                \item Scan des systèmes internes au SI
                \item Facilitation d'autres attaques
			\end{itemize}
			\vfill
		\end{column}
		\begin{column}{0.495\textwidth}
			\vfill
			\uline{\textbf{Prévention}}
			\begin{itemize}
				\item Utiliser du JSON plutôt que du XML
                \item Mettre à jour les librairies et processeurs XML
                \item Désactiver les entités xml externes et le DTD
                \item Implementer une whitelist coté serveur pour empêcher
                    l'injection de XML hostile.
			\end{itemize}
		\end{column}
	\end{columns}
\end{block}


%----------------------------------------------------------------------------------------
%	METHODS
%----------------------------------------------------------------------------------------


%----------------------------------------------------------------------------------------
%	MATHEMATICAL SECTION
%----------------------------------------------------------------------------------------

\begin{block}{Mathematical Section}

\begin{itemize}
\item Maecenas Ultricies Feugiat Velit Non Mattis.
\begin{itemize}
\item Duis ante erat, bibendum nec tempus nec, interdum quis est. Nulla at mollis tortor. Phasellus quis leo dolor, aliquam laoreet orci $X$ Donec dapibus sagittis neque eu nec, interdum quis est. $Y_n, n=1,\cdots,N$ ndum nec tempus nec, interd
\begin{align*}
X \rightarrow r(X) & = \arg \max_{c} \Big\{ \max_n \big\{ \sum_{x_i \in X} \delta(x_i,Y_{n,c})\big\} \Big\} 
\end{align*}
\item Cras faucibus scelerisque cursus. Proin ut vestibulum augue. $\delta(x_i,Y_{n,c})$
\end{itemize}
\item Fusce tempus arcu id ligula varius dictum. Donec ut nisl dui, ac consectetur elit. In nec enim porta augue venenatis sollicitudin. Phasellus quis nunc neque. Suspendisse mauris diam, suscipit non gravida in, placerat id enim. Ut nec ipsum in lectus ultrices sagittis.
\end{itemize}

\end{block}

%----------------------------------------------------------------------------------------

\end{column} % End of the first column

\begin{column}{.03\textwidth}\end{column} % Empty spacer column
 
\begin{column}{.465\textwidth} % The second column

%----------------------------------------------------------------------------------------
%	RESULTS
%----------------------------------------------------------------------------------------

\begin{block}{Results: Table}

\begin{itemize}
\item Ased Aliquet Luctus Lectus
\end{itemize}

\begin{table}
\begin{tabular}{l l l}
\toprule
\textbf{Treatments} & \textbf{Response 1} & \textbf{Response 2}\\
\midrule
Treatment 1 & 0.0003262 & 0.562 \\
Treatment 2 & 0.0015681 & 0.910 \\
Treatment 3 & 0.0009271 & 0.296 \\
\bottomrule
\end{tabular}
\caption{Table caption}
\end{table}

\begin{itemize}
\item Sollicitudin Vel Orci
\item Maecenas Ultricies Feugiat Velit Non Mattis.
\end{itemize}

\begin{table}
\begin{tabular}{l l l}
\toprule
\textbf{Treatments} & \textbf{Response 1} & \textbf{Response 2}\\
\midrule
Treatment 1 & 0.0003262 & 0.562 \\
Treatment 2 & 0.0015681 & 0.910 \\
Treatment 3 & 0.0009271 & 0.296 \\
\bottomrule
\end{tabular}
\caption{Table caption}
\end{table}
     
\end{block}

%------------------------------------------------

\begin{block}{Results: Figure}

\begin{figure}
\includegraphics[width=0.8\linewidth]{placeholder.jpg}
\caption{Figure caption}
\end{figure}

\end{block}

%----------------------------------------------------------------------------------------
%	CONCLUSION
%----------------------------------------------------------------------------------------

\begin{block}{Conclusion}

\begin{itemize}
\item Opet volutpat ligula. Duis semper lorem eget dui dignissim porttitor. Nulla facilisi. In ullamcorper lorem quis dolor iaculis nec egestas enim ultricies. Cras ut mauris elit, ut lacinia dui. Proin in ante et libero hendrerit iaculis.
\item Nulla eu erat a urna laoreet auctor id a turpis. Nam mollis tristique neque eu luctus. Suspendisse rutrum congue nisi sed convallis. 
\item Aenean id neque dolor.
\item Opet volutpat ligula. Duis semper lorem eget dui dignissim porttitor. Nulla facilisi. In ullamcorper lorem quis dolor iaculis nec egestas enim ultricies. Cras ut mauris elit, ut lacinia dui. Proin in ante et libero hendrerit iaculis.
\end{itemize}

\end{block}

%----------------------------------------------------------------------------------------
%	REFERENCES
%----------------------------------------------------------------------------------------

\begin{block}{References}
        
\nocite{*} % Insert publications even if they are not cited in the poster
\small{\bibliographystyle{unsrt}
\bibliography{sample}}

\end{block}

%----------------------------------------------------------------------------------------
%	ACKNOWLEDGEMENTS
%----------------------------------------------------------------------------------------

\begin{block}{Acknowledgments}

\begin{itemize}
\item Nam mollis tristique neque eu luctus. Suspendisse rutrum congue nisi sed convallis. Aenean id neque dolor. Pellentesque habitant morbi tristique senectus et netus et malesuada fames ac turpis egestas.
\end{itemize}

\end{block}

%----------------------------------------------------------------------------------------
%	CONTACT INFORMATION
%----------------------------------------------------------------------------------------

\setbeamercolor{block title}{fg=black,bg=orange!70} % Change the block title color

\begin{block}{Contact}

\begin{itemize}
\item Web: \href{https://www.serli.com}{https://www.serli.com}
\item Email: \href{mailto:quentin.mallet@serli.com}{quentin.mallet@serli.com}
\end{itemize}

\end{block}

%----------------------------------------------------------------------------------------

\end{column} % End of the second column

\begin{column}{.015\textwidth}\end{column} % Empty spacer column

\end{columns} % End of all the columns in the poster

\end{frame} % End of the enclosing frame

\end{document}
